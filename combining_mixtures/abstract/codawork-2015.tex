%* EXAMPLE FOR CODAWORK'15 *********


\documentclass [10pt]{article}

%* some useful packages *********
\usepackage{amsfonts,amssymb}
%\usepackage[maxcitenames=2]{biblatex}
\usepackage{epsfig,chicago,float}
%\usepackage[maxnames=2]{chicago}
\usepackage[utf8]{inputenc}

%****************************************************************
\setlength{\oddsidemargin}{+4.6mm}
%
\setlength{\textwidth}{15cm}
%
\setlength{\textheight}{23cm}
%
\setlength{\topmargin}{-1.25cm} \setlength{\baselineskip}{1mm}
%
\setlength{\parindent}{0pt}
%
\setlength{\parskip}{0.25cm}
%
\pagestyle{empty}
%
\renewcommand{\refname}{\centerline{REFERENCES}}
%
%*for Spanish*************************************************
\newcommand{\enye}{\~n}
%**************************************************************


%*for captions*******************************************
\renewcommand{\figurename}{\footnotesize{\bf Figure}}
\renewcommand{\tablename}{\footnotesize{\bf Table}}
%*****************************************************


%****************the document **************************

\begin{document}
\begin{center}
\textbf{\large A compositional approach for merging finite mixture components}

\vskip 0.25cm

\textbf{M. Comas-Cufí}$^{1}$\textbf{, J.A. Martín-Fernández}$^{1}$\textbf{, and
G. Mateu-Figueras}$^{1}$ \\
{\small $^{1}$Universitat de Girona, Girona, Catalonia;
\textit{mcomas@imae.udg.edu}}
\end{center}

\vskip 0.5cm {\centerline{\bf Abstract}}

The particular case of merging the components of a Gaussian mixture hierarchically has received special attention by different authors (e.g., Hennig (2010) reviewed different approaches). In this work, we focus on those approaches that rely on the posterior probabilities $\tau_{ij}$, that is, the posterior probability of observation $i$, $1 \leq i \leq n$ being generated by component $j$, $1\leq j\leq k$. These components are fitted after adjusting a  probability distribution mixture with $k$ components  to a sample $X$ with $n$ observations (Melnykov, 2013; Hennig, 2010; Baudry and others, 2010; Comas-Cufí and others, 2013).  Although this approaches have been widely developed for Gaussian mixtures, they can be applied for any type of probability distribution mixture.

In this work, different approaches based on posterior probabilities are integrated in a single formulation. Using this new formulation, the main differences between the reviewed approaches are analysed and discussed. Furthermore, we will see that the new formulation allows to naturally introduce new techniques for combining the mixture components hierarchically.  An experiment using simulated data is presented to illustrate and compare the performance of the different approaches.

\vskip 0.2cm {\bf References}

\vspace{0.5mm}
\hangindent=0.5cm \hangafter=1 %
Baudry, J.-P., A.~E. Raftery, G.~Celeux, K.~Lo, and R.~Gottardo (2010).
\newblock {Combining Mixture Components for Clustering}.
\newblock {\em Journal of Computational and Graphical Statistics\/}~{\em 9\/}(2), pp. 332--353.

\hangindent=0.5cm \hangafter=1 %
Comas-Cuf\'{\i}, M., J.~A. Mart\'{\i}n-Fern\'{a}ndez, and G.~Mateu-Figueras
  (2013).
\newblock {Compositional entropies in model based clustering}.
\newblock In {\em 6th International Conference of the ERCIM (European Research
  Consortium for Informatics and Mathematics) Working Group on Computational
  and Methodological Statistics (ERCIM 2013)}, London (UK), pp.\  122.

\hangindent=0.5cm \hangafter=1 %
Hennig, C. (2010).
\newblock {Methods for merging Gaussian mixture components}.
\newblock {\em Advances in Data Analysis and Classification\/}~{\em 4\/}(1), pp.  3--34.

\hangindent=0.5cm \hangafter=1 %
Melnykov, V. (2013).
\newblock {On the Distribution of Posterior Probabilities in Finite Mixture
  Models with Application in Clustering}.
\newblock {\em Journal of Multivariate Analysis\/}~{\em 122}, pp. 175--189.



\end{document}