\begin{frame}[t]
\frametitle{Where do $\boldsymbol\tau_i$'s come from?}

\begin{description}
\item[Mixture models]  The a posteriori probabilities
\begin{eqnarray*} \tau_{ij} &=& \frac{\pi_j f(x_i; \theta_j)}{\sum_{\ell = 1}^k \pi_\ell f(x_i; \theta_\ell) } \end{eqnarray*}
\item[Expert decision] An expert set the weights according to its expertise
\item[Fuzzy clustering] The membership weight \uncover<2->{ (if $f'(x; C) := \frac{1}{d(x, C)}^{2/(m-1)}$)  }
\begin{eqnarray*}
\tau_{ij} &=& \frac{1}{\sum_{\ell = 1}^k \left( \frac{d(x_i; C_j)}{d(x_i; C_\ell) } \right)^{2/(m-1)}}  \\
\uncover<3>{ &=& \frac{f'(x_i; C_j)}{\sum_{\ell = 1}^k f'(x_i; C_\ell)} }
\end{eqnarray*}
\end{description}
\end{frame}



\begin{frame}[T]
\frametitle{Whats does "merge" usually mean  in general?}
%\frametitle{Merging operation}
\small
\begin{itemize}
\item In mixture models  we have used the amalgamation, $\tau_{ic} = \tau_{ia} + \tau_{ib}$.
\end{itemize}
\begin{columns}[t]
\column{0.5\textwidth}

\textbf{Sequentially}
\begin{itemize}
\item Maximum \[\tau_{ic} =  max\left\{ \tau_{ia},  \tau_{ib} \right\} \]
\item Product \[\tau_{ic} =  \tau_{ia}  \tau_{ib} \]
\item Geometric mean \[\tau_{ic} = \sqrt{ \tau_{ia}  \tau_{ib} }\]
\end{itemize}
\column{0.5\textwidth}

\textbf{Within class}
\begin{itemize}
\item Maximum \[\tau_{iC} =  max_{c \in C}   \tau_{ic} \]
\item Product \[\tau_{iC} =  \prod_{c \in C}  \tau_{ic}  \]
\item Geometric mean \[\tau_{iC} = \left( \prod_{c \in C} \tau_{i} \right)^{1/|C|}  \]
\end{itemize}
\end{columns}

\end{frame}


\begin{frame}[t, label=first-aliments]
\frametitle{Protein consumption Europe }
\scriptsize
\begin{table}[ht]
\begin{center}
\begin{tabular}{rrrrrrrrrr}
  \hline
 & RM & WM & E & M & F & C & S & N & FV \\
  \hline
AL & 14.19 & 1.97 & 0.70 & 12.50 & 0.28 & 59.41 & 0.84 & 7.72 & 2.39 \\
  AT & 10.30 & 16.20 & 4.98 & 23.03 & 2.43 & 32.41 & 4.17 & 1.50 & 4.98 \\
  BE & 15.46 & 10.65 & 4.70 & 20.05 & 5.15 & 30.47 & 6.53 & 2.41 & 4.58 \\
\vdots&\vdots&\vdots&\vdots&\vdots&\vdots&\vdots&\vdots&\vdots&\vdots\\
  SU & 10.12 & 5.01 & 2.29 & 18.06 & 3.26 & 47.44 & 6.96 & 3.70 & 3.16 \\
  WD & 14.38 & 15.76 & 5.17 & 23.71 & 4.29 & 23.46 & 6.56 & 1.89 & 4.79 \\
  YG & 4.97 & 5.65 & 1.36 & 10.73 & 0.68 & 63.16 & 3.39 & 6.44 & 3.62 \\
   \hline
\end{tabular}
\end{center}
\end{table}
\tiny Protein consumption percentage comming from different source.
RM: red meat, WM: white meat, E: eggs, M: milk, F: fish, C: cereals, S: starch, N: nuts, FV: fruits and vegetables. \\EUROSTAT, 1999

\medskip
\small
\begin{itemize}
\item Merging approach: $\tau_{iC} = \left( \prod_{c \in C} \tau_{i} \right)^{1/|C|}$ (geometric mean)
\item $\varphi(\boldsymbol\tau_i, a) = \tau_{ia}$
\item $\omega(\boldsymbol\tau_i, a, b) = \log \tau_{ia}/\tau_{ib}$
\end{itemize}
\hyperlink{last-aliments<1>}{\beamergotobutton{Skip example}}
\end{frame}

\begin{frame}[fragile, t]
\frametitle{Protein consumption Europe \hyperlink{last-aliments<1>}{\beamergotobutton{Skip example}}}
\scriptsize
\begin{verbatim}
# partitions = 25
-> Albania
-> Austria
-> Belgium
-> Bulgaria
-> Czechoslovakia
-> Danemark
-> EastGermany
-> Finland
-> France
-> Greece
-> Hungary
-> Ireland
-> Italy
-> Netherlands
-> Norway
-> Poland
-> Portugal
-> Romania
-> Spain
-> Sweden
-> Switzerland
-> UnitedKingdrom
-> SovietUnion
-> WestGermany
-> Yugoslavia
\end{verbatim}

\end{frame}
\begin{frame}[fragile, t]
\frametitle{Protein consumption Europe \hyperlink{last-aliments<1>}{\beamergotobutton{Skip example}}}
\scriptsize
\begin{verbatim}
# partitions = 24
-> Albania
-> Austria
-> EastGermany | Belgium
-> Bulgaria
-> Czechoslovakia
-> Danemark
-> Finland
-> France
-> Greece
-> Hungary
-> Ireland
-> Italy
-> Netherlands
-> Norway
-> Poland
-> Portugal
-> Romania
-> Spain
-> Sweden
-> Switzerland
-> UnitedKingdrom
-> SovietUnion
-> WestGermany
-> Yugoslavia
\end{verbatim}

\end{frame}
\begin{frame}[fragile, t]
\frametitle{Protein consumption Europe \hyperlink{last-aliments<1>}{\beamergotobutton{Skip example}}}
\scriptsize
\begin{verbatim}
# partitions = 23
-> Albania
-> Austria
-> Bulgaria
-> Czechoslovakia
-> Danemark
-> Finland
-> France
-> Greece
-> Hungary
-> Ireland
-> Italy
-> Netherlands
-> Norway
-> Poland
-> Portugal
-> Romania
-> Spain
-> Sweden
-> Switzerland
-> UnitedKingdrom
-> SovietUnion
-> EastGermany | Belgium | WestGermany
-> Yugoslavia
\end{verbatim}

\end{frame}
\begin{frame}[fragile, t]
\frametitle{Protein consumption Europe \hyperlink{last-aliments<1>}{\beamergotobutton{Skip example}}}
\scriptsize
\begin{verbatim}
# partitions = 22
-> Albania
-> Austria
-> Bulgaria
-> Czechoslovakia
-> Danemark
-> Finland
-> France
-> Hungary
-> Ireland
-> Greece | Italy
-> Netherlands
-> Norway
-> Poland
-> Portugal
-> Romania
-> Spain
-> Sweden
-> Switzerland
-> UnitedKingdrom
-> SovietUnion
-> EastGermany | Belgium | WestGermany
-> Yugoslavia
\end{verbatim}

\end{frame}
\begin{frame}[fragile, t]
\frametitle{Protein consumption Europe \hyperlink{last-aliments<1>}{\beamergotobutton{Skip example}}}
\scriptsize
\begin{verbatim}
# partitions = 21
-> Albania
-> Austria
-> Bulgaria
-> Czechoslovakia
-> Danemark
-> Finland
-> France
-> Hungary
-> Ireland
-> Greece | Italy
-> Netherlands
-> Norway
-> Poland
-> Portugal
-> Yugoslavia | Romania
-> Spain
-> Sweden
-> Switzerland
-> UnitedKingdrom
-> SovietUnion
-> EastGermany | Belgium | WestGermany
\end{verbatim}

\end{frame}
\begin{frame}[fragile, t]
\frametitle{Protein consumption Europe \hyperlink{last-aliments<1>}{\beamergotobutton{Skip example}}}
\scriptsize
\begin{verbatim}
# partitions = 20
-> Albania
-> Austria
-> Bulgaria
-> Czechoslovakia
-> Finland
-> France
-> Hungary
-> Ireland
-> Greece | Italy
-> Netherlands
-> Norway
-> Poland
-> Portugal
-> Yugoslavia | Romania
-> Spain
-> Danemark | Sweden
-> Switzerland
-> UnitedKingdrom
-> SovietUnion
-> EastGermany | Belgium | WestGermany
\end{verbatim}

\end{frame}
\begin{frame}[fragile, t]
\frametitle{Protein consumption Europe \hyperlink{last-aliments<1>}{\beamergotobutton{Skip example}}}
\scriptsize
\begin{verbatim}
# partitions = 19
-> Albania
-> Bulgaria
-> Czechoslovakia
-> Finland
-> France
-> Hungary
-> Ireland
-> Greece | Italy
-> Austria | Netherlands
-> Norway
-> Poland
-> Portugal
-> Yugoslavia | Romania
-> Spain
-> Danemark | Sweden
-> Switzerland
-> UnitedKingdrom
-> SovietUnion
-> EastGermany | Belgium | WestGermany
\end{verbatim}

\end{frame}
\begin{frame}[fragile, t]
\frametitle{Protein consumption Europe \hyperlink{last-aliments<1>}{\beamergotobutton{Skip example}}}
\scriptsize
\begin{verbatim}
# partitions = 18
-> Albania
-> Bulgaria
-> Czechoslovakia
-> Finland
-> France
-> Hungary
-> Ireland
-> Greece | Italy
-> Austria | Netherlands
-> Norway
-> Poland
-> Portugal
-> Yugoslavia | Romania
-> Spain
-> Switzerland
-> UnitedKingdrom
-> SovietUnion
-> Danemark | Sweden | EastGermany | Belgium | WestGermany
\end{verbatim}

\end{frame}
\begin{frame}[fragile, t]
\frametitle{Protein consumption Europe \hyperlink{last-aliments<1>}{\beamergotobutton{Skip example}}}
\scriptsize
\begin{verbatim}
# partitions = 17
-> Albania
-> Bulgaria
-> Czechoslovakia
-> Finland
-> Danemark | Sweden | EastGermany | Belgium | WestGermany | France
-> Hungary
-> Ireland
-> Greece | Italy
-> Austria | Netherlands
-> Norway
-> Poland
-> Portugal
-> Yugoslavia | Romania
-> Spain
-> Switzerland
-> UnitedKingdrom
-> SovietUnion
\end{verbatim}

\end{frame}
\begin{frame}[fragile, t]
\frametitle{Protein consumption Europe \hyperlink{last-aliments<1>}{\beamergotobutton{Skip example}}}
\scriptsize
\begin{verbatim}
# partitions = 16
-> Albania
-> Bulgaria
-> Czechoslovakia
-> Finland
-> Hungary
-> Ireland
-> Greece | Italy
-> Danemark | Sweden | EastGermany | Belgium | WestGermany | France | Austria | 
Netherlands
-> Norway
-> Poland
-> Portugal
-> Yugoslavia | Romania
-> Spain
-> Switzerland
-> UnitedKingdrom
-> SovietUnion
\end{verbatim}

\end{frame}
\begin{frame}[fragile, t]
\frametitle{Protein consumption Europe \hyperlink{last-aliments<1>}{\beamergotobutton{Skip example}}}
\scriptsize
\begin{verbatim}
# partitions = 15
-> Albania
-> Bulgaria
-> Czechoslovakia
-> Finland
-> Hungary
-> Danemark | Sweden | EastGermany | Belgium | WestGermany | France | Austria | 
Netherlands | Ireland
-> Greece | Italy
-> Norway
-> Poland
-> Portugal
-> Yugoslavia | Romania
-> Spain
-> Switzerland
-> UnitedKingdrom
-> SovietUnion
\end{verbatim}

\end{frame}
\begin{frame}[fragile, t]
\frametitle{Protein consumption Europe \hyperlink{last-aliments<1>}{\beamergotobutton{Skip example}}}
\scriptsize
\begin{verbatim}
# partitions = 14
-> Albania
-> Bulgaria
-> Czechoslovakia
-> Finland
-> Hungary
-> Greece | Italy
-> Norway
-> Poland
-> Portugal
-> Yugoslavia | Romania
-> Spain
-> Danemark | Sweden | EastGermany | Belgium | WestGermany | France | Austria | 
Netherlands | Ireland | Switzerland
-> UnitedKingdrom
-> SovietUnion
\end{verbatim}

\end{frame}
\begin{frame}[fragile, t]
\frametitle{Protein consumption Europe \hyperlink{last-aliments<1>}{\beamergotobutton{Skip example}}}
\scriptsize
\begin{verbatim}
# partitions = 13
-> Albania
-> Bulgaria
-> Danemark | Sweden | EastGermany | Belgium | WestGermany | France | Austria | 
Netherlands | Ireland | Switzerland | Czechoslovakia
-> Finland
-> Hungary
-> Greece | Italy
-> Norway
-> Poland
-> Portugal
-> Yugoslavia | Romania
-> Spain
-> UnitedKingdrom
-> SovietUnion
\end{verbatim}

\end{frame}
\begin{frame}[fragile, t]
\frametitle{Protein consumption Europe \hyperlink{last-aliments<1>}{\beamergotobutton{Skip example}}}
\scriptsize
\begin{verbatim}
# partitions = 12
-> Albania
-> Bulgaria
-> Finland
-> Hungary
-> Greece | Italy
-> Norway
-> Danemark | Sweden | EastGermany | Belgium | WestGermany | France | Austria | 
Netherlands | Ireland | Switzerland | Czechoslovakia | Poland
-> Portugal
-> Yugoslavia | Romania
-> Spain
-> UnitedKingdrom
-> SovietUnion
\end{verbatim}

\end{frame}
\begin{frame}[fragile, t]
\frametitle{Protein consumption Europe \hyperlink{last-aliments<1>}{\beamergotobutton{Skip example}}}
\scriptsize
\begin{verbatim}
# partitions = 11
-> Albania
-> Bulgaria
-> Hungary
-> Greece | Italy
-> Finland | Norway
-> Danemark | Sweden | EastGermany | Belgium | WestGermany | France | Austria | 
Netherlands | Ireland | Switzerland | Czechoslovakia | Poland
-> Portugal
-> Yugoslavia | Romania
-> Spain
-> UnitedKingdrom
-> SovietUnion
\end{verbatim}

\end{frame}
\begin{frame}[fragile, t]
\frametitle{Protein consumption Europe \hyperlink{last-aliments<1>}{\beamergotobutton{Skip example}}}
\scriptsize
\begin{verbatim}
# partitions = 10
-> Albania
-> Bulgaria
-> Hungary
-> Greece | Italy
-> Danemark | Sweden | EastGermany | Belgium | WestGermany | France | Austria | 
Netherlands | Ireland | Switzerland | Czechoslovakia | Poland
-> Portugal
-> Yugoslavia | Romania
-> Spain
-> Finland | Norway | UnitedKingdrom
-> SovietUnion
\end{verbatim}

\end{frame}
\begin{frame}[fragile, t]
\frametitle{Protein consumption Europe \hyperlink{last-aliments<1>}{\beamergotobutton{Skip example}}}
\scriptsize
\begin{verbatim}
# partitions =  9
-> Albania
-> Bulgaria
-> Hungary
-> Greece | Italy
-> Finland | Norway | UnitedKingdrom | Danemark | Sweden | EastGermany | Belgium | 
WestGermany | France | Austria | Netherlands | Ireland | Switzerland | 
Czechoslovakia | Poland
-> Portugal
-> Yugoslavia | Romania
-> Spain
-> SovietUnion
\end{verbatim}

\end{frame}
\begin{frame}[fragile, t]
\frametitle{Protein consumption Europe \hyperlink{last-aliments<1>}{\beamergotobutton{Skip example}}}
\scriptsize
\begin{verbatim}
# partitions =  8
-> Albania
-> Bulgaria
-> Hungary
-> Greece | Italy
-> Finland | Norway | UnitedKingdrom | Danemark | Sweden | EastGermany | Belgium | 
WestGermany | France | Austria | Netherlands | Ireland | Switzerland | 
Czechoslovakia | Poland
-> Yugoslavia | Romania
-> Portugal | Spain
-> SovietUnion
\end{verbatim}

\end{frame}
\begin{frame}[fragile, t]
\frametitle{Protein consumption Europe \hyperlink{last-aliments<1>}{\beamergotobutton{Skip example}}}
\scriptsize
\begin{verbatim}
# partitions =  7
-> Albania
-> Bulgaria
-> Hungary
-> Greece | Italy
-> Finland | Norway | UnitedKingdrom | Danemark | Sweden | EastGermany | Belgium | 
WestGermany | France | Austria | Netherlands | Ireland | Switzerland | 
Czechoslovakia | Poland
-> Yugoslavia | Romania
-> Portugal | Spain | SovietUnion
\end{verbatim}

\end{frame}
\begin{frame}[fragile, t]
\frametitle{Protein consumption Europe \hyperlink{last-aliments<1>}{\beamergotobutton{Skip example}}}
\scriptsize
\begin{verbatim}
# partitions =  6
-> Albania
-> Bulgaria
-> Hungary
-> Portugal | Spain | SovietUnion | Greece | Italy
-> Finland | Norway | UnitedKingdrom | Danemark | Sweden | EastGermany | Belgium | 
WestGermany | France | Austria | Netherlands | Ireland | Switzerland | 
Czechoslovakia | Poland
-> Yugoslavia | Romania
\end{verbatim}

\end{frame}
\begin{frame}[fragile, t]
\frametitle{Protein consumption Europe \hyperlink{last-aliments<1>}{\beamergotobutton{Skip example}}}
\scriptsize
\begin{verbatim}
# partitions =  5
-> Albania
-> Bulgaria
-> Hungary
-> Portugal | Spain | SovietUnion | Greece | Italy | Finland | Norway | 
UnitedKingdrom | Danemark | Sweden | EastGermany | Belgium | WestGermany | France | 
Austria | Netherlands | Ireland | Switzerland | Czechoslovakia | Poland
-> Yugoslavia | Romania
\end{verbatim}

\end{frame}
\begin{frame}[fragile, t]
\frametitle{Protein consumption Europe \hyperlink{last-aliments<1>}{\beamergotobutton{Skip example}}}
\scriptsize
\begin{verbatim}
# partitions =  4
-> Albania
-> Bulgaria
-> Portugal | Spain | SovietUnion | Greece | Italy | Finland | Norway | 
UnitedKingdrom | Danemark | Sweden | EastGermany | Belgium | WestGermany | France | 
Austria | Netherlands | Ireland | Switzerland | Czechoslovakia | Poland
-> Hungary | Yugoslavia | Romania
\end{verbatim}

\end{frame}
\begin{frame}[fragile, t]
\frametitle{Protein consumption Europe \hyperlink{last-aliments<1>}{\beamergotobutton{Skip example}}}
\scriptsize
\begin{verbatim}
# partitions =  3
-> Albania
-> Hungary | Yugoslavia | Romania | Bulgaria
-> Portugal | Spain | SovietUnion | Greece | Italy | Finland | Norway | 
UnitedKingdrom | Danemark | Sweden | EastGermany | Belgium | WestGermany | France | 
Austria | Netherlands | Ireland | Switzerland | Czechoslovakia | Poland
\end{verbatim}

\end{frame}
\begin{frame}[fragile, t]
\frametitle{Protein consumption Europe \hyperlink{last-aliments<1>}{\beamergotobutton{Skip example}}}
\scriptsize
\begin{verbatim}
# partitions =  2
-> Albania
-> Hungary | Yugoslavia | Romania | Bulgaria | Portugal | Spain | SovietUnion | 
Greece | Italy | Finland | Norway | UnitedKingdrom | Danemark | Sweden | 
EastGermany | Belgium | WestGermany | France | Austria | Netherlands | Ireland | 
Switzerland | Czechoslovakia | Poland
\end{verbatim}

\end{frame}
\begin{frame}[fragile, t, label=last-aliments]
\frametitle{Protein consumption Europe \hyperlink{first-aliments<1>}{\beamergotobutton{Go to first}}}
\scriptsize
\begin{verbatim}
# partitions =  1
-> Albania | Hungary | Yugoslavia | Romania | Bulgaria | Portugal | Spain | 
SovietUnion | Greece | Italy | Finland | Norway | UnitedKingdrom | Danemark | 
Sweden | EastGermany | Belgium | WestGermany | France | Austria | Netherlands | 
Ireland | Switzerland | Czechoslovakia | Poland
\end{verbatim}

\end{frame}