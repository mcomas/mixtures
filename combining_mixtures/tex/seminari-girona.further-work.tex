\begin{frame}[t]
\frametitle{Where do $\boldsymbol\tau_i$'s come from?}

\begin{description}
\item[Mixture models]  The a posteriori probabilities
\begin{eqnarray*} \tau_{ij} &=& \frac{\pi_j f(x_i; \theta_j)}{\sum_{\ell = 1}^k \pi_\ell f(x_i; \theta_\ell) } \end{eqnarray*}
\item[Expert decision] An expert set the weights according to its expertise
\item[Fuzzy clustering] The membership weight \uncover<2->{ (if $f'(x; C) := \frac{1}{d(x, C)}^{2/(m-1)}$)  }
\begin{eqnarray*}
\tau_{ij} &=& \frac{1}{\sum_{\ell = 1}^k \left( \frac{d(x_i; C_j)}{d(x_i; C_\ell) } \right)^{2/(m-1)}}  \\
\uncover<3>{ &=& \frac{f'(x_i; C_j)}{\sum_{\ell = 1}^k f'(x_i; C_\ell)} }
\end{eqnarray*}
\end{description}
\end{frame}



\begin{frame}[T]
\frametitle{Whats does "merge" usually mean  in general?}
%\frametitle{Merging operation}
\small
\begin{itemize}
\item In mixture models  we have used the amalgamation, $\tau_{ic} = \tau_{ia} + \tau_{ib}$.
\end{itemize}
\begin{columns}[t]
\column{0.5\textwidth}

\textbf{Sequentially}
\begin{itemize}
\item Maximum \[\tau_{ic} =  max\left\{ \tau_{ia},  \tau_{ib} \right\} \]
\item Product \[\tau_{ic} =  \tau_{ia}  \tau_{ib} \]
\item Geometric mean \[\tau_{ic} = \sqrt{ \tau_{ia}  \tau_{ib} }\]
\end{itemize}
\column{0.5\textwidth}

\textbf{Within class}
\begin{itemize}
\item Maximum \[\tau_{iC} =  max_{c \in C}   \tau_{ic} \]
\item Product \[\tau_{iC} =  \prod_{c \in C}  \tau_{ic}  \]
\item Geometric mean \[\tau_{iC} = \left( \prod_{c \in C} \tau_{i} \right)^{1/|C|}  \]
\end{itemize}
\end{columns}

\end{frame}
