% Explicació genèrica del problema a tractar
\begin{frame}[t]
%Estaria bé tenir un dibuix on es veièssin les classes a dalt (pensar quina forma podrien tenir), les observacions a baix sense que estiguin necessariament ordenades i un vector de probabilitats de pertanyer a cada una de les classes.
\frametitle{Weights or probabilities related to a class}

\begin{itemize}
\item A sample $\obs{\textbf{x}_1}, \dots, \obs{\textbf{x}_n}$
\item Classes $\class{C_1}, \dots, \class{C_k}$
\item<3-> For each $\obs{\textbf{x}_i}$ a vector of probabilities $\post{\tau_{i1}}, \dots, \post{\tau_{ik}}$ giving the relative chances to belong to certain classes $\class{C_1}, \dots, \class{C_k}$
\end{itemize}

\bigskip
\begin{center}
\input{tex/seminari-girona.fig.observation-posteriori-diagram.tex}
\end{center}
\end{frame}


\begin{frame}[t]
\frametitle{Partitioning a finite set}
\begin{block}{Partition of a finite set}
Let $I^k = \{1, \dots, k \}$ be a finite set. A \emph{partition of $I^k$}  is a set $\mathcal{I}$ of subsets of $I^k$, each subset  called part, such that 
\begin{itemize}
\item $\bigcup_{I \in \mathcal{I}} = I^k$ and 
\item $I \cap J = \emptyset$  for every $I, J \in \mathcal{I}$ different.
\end{itemize}
\end{block}

\medskip
\pause

\begin{columns}
\column{0.5\textwidth}%
For example if $I^4$
\begin{itemize}
\item $\mathcal{I} = \{ \{1, 2, 3, 4\} \}$
\item $\mathcal{I} = \{ \{1 \}, \{2, 4\}, 3 \}$
\item $\mathcal{I} = \{ \{1\}, \{2\}, \{3\}, \{4\} \}$
\end{itemize}
\column{0.5\textwidth}
\only<4>{%
For example if $F = \{H, O, L, A\}$
\begin{itemize}
\item $\mathcal{I} = \{ \{H, O, L, A \} \}$
\item $\mathcal{I} = \{ \{H \}, \{O, A \}, L \}$
\item $\mathcal{I} = \{ \{H\}, \{O \}, \{L \}, \{A \} \}$
\end{itemize}
}%
\end{columns}

\medskip
\pause
\begin{itemize}
\item[] For any finite set $F$, the partition can be defined naturally indexing its elements.
\end{itemize}
\end{frame}

\begin{frame}[t]
\frametitle{Hierarchical combination of a finite set}
\begin{block}{Hierarchical combination of a finite set}
A hierarchical combination of a finite set with $k$ elements is a sequence of partitions $\mathcal{I}_1, \mathcal{I}_2, \dots \mathcal{I}_k$ of $I^k$, such that for each i, $1 \leq i \leq k$,
\begin{itemize}
\item $\mathcal{I}_i$ has $i$ elements and
\item if a part $I \in \mathcal{I}_{i-1}$ then there is a part $J \in \mathcal{I}_i$ with $J = I$ or there are two parts $J_1, J_2 \in \mathcal{I}_i$ with $I = J_1 \cup J_2$.
\end{itemize}
\end{block}

For example,
\begin{itemize}
\item A hierarchical combination of a finite set defines a binary partition of the finite set.
\end{itemize}

\end{frame}

% Especificació del problema a resoldre.
\begin{frame}[t]
\frametitle{Our input, our goal: What do we want to do?}

\begin{block}{Input}
A sample of probabilities to belong to some classe $C_i$, 
\begin{columns}
\column{0.4\textwidth}
\[ T = \left[ \begin{array}{ccccc}
\idea{(}\tau_{11}\idea{,} & \dots & \tau_{1j}\idea{,} & \dots & \tau_{1k}\idea{),} \\
\vdots      & &    \vdots                     & &    \vdots                     \\
\idea{(}\tau_{11}\idea{,} & \dots & \tau_{1j}\idea{,} & \dots & \tau_{1k}\idea{),} \\
\vdots      & &      \vdots                   & &       \vdots                  \\
\idea{(}\tau_{11}\idea{,} & \dots & \tau_{1j}\idea{,} & \dots & \tau_{1k}\idea{)}
\end{array} \right] 
\idea{ \in \mathcal{S}_{[C_1,\dots,C_k]}^k } \]
\column{0.3\textwidth}
\end{columns}
\end{block}

\pause
\begin{alertblock}{Goal}
\alert{Merge} classes (sequentially) to obtain a hierarchy over the set of classes % $\{C_1, \dots, C_k\}$. %In other words, obtain a binary tree with a set of leafs $\{C_1, \dots, C_k\}$
\end{alertblock}
\end{frame}




\begin{frame}[t]
\frametitle{Where do $\boldsymbol\tau_i$'s come from?}

\begin{description}
\item[Expert decision] An expert set the weights according to its expertise
\item[Mixture models]  The a posteriori probabilities
\begin{eqnarray*} \tau_{ij} &=& \frac{f(x_i; \theta_j)}{\sum_{\ell = 1}^k f(x_i; \theta_\ell) } \end{eqnarray*}
\item[Fuzzy clustering] The membership weight \uncover<2->{ (if $f'(x; C) := \frac{1}{d(x, C)}^{2/(m-1)}$)  }
\begin{eqnarray*}
\tau_{ij} &=& \frac{1}{\sum_{\ell = 1}^k \left( \frac{d(x_i; C_j)}{d(x_i; C_\ell) } \right)^{2/(m-1)}}  \\
\uncover<3>{ &=& \frac{f'(x_i; C_j)}{\sum_{\ell = 1}^k f'(x_i; C_\ell)} }
\end{eqnarray*}
\end{description}
\end{frame}


