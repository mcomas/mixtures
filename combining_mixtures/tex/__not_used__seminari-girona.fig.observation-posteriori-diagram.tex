\begin{tikzpicture}[scale=0.5]
  
  %\tikzstyle{obs}   = [minimum size=1.1em]
  \tikzstyle{class} = [circle, draw, fill=class!20, minimum size=1em]
  %\tikzstyle{post}  = [ minimum size=1.8em]
  \tikzstyle{line}  = [draw, thick, -latex']

   \newcommand*{\horiz}{20}
   \newcommand*{\verti}{8}
   
   \newcommand*{\lobs}{7}
   \newcommand*{\lpost}{-1}
   \newcommand*{\osep}{14.5}
   \newcommand*{\psep}{4}

  %\draw[help lines] (0,0) grid (\horiz,\verti);
  \node<2-> (C1) [class] at(2, 5) {\tiny $C_1$};
  \node<2-> (C2) [class] at(7, 5.5) {\tiny $C_2$};
  \node<2-> (C3) [class] at(11.5, 2.25) {\tiny $C_3$};
  \node<2-> (C4) [class] at(3, 1) {\tiny $C_4$};
  
  \node<1-3> (Xi) [obs] at (7,4) {\scriptsize $x_i$};
  \node<1-4> (X2) [obs] at (2,2.5) {\scriptsize $x_2$};
  \node<1-4> (X3) [obs] at (4,2) {\scriptsize $x_3$};
  \node<1-4> (X4) [obs] at (1,4) {\scriptsize $x_4$};
  \node<1-4> (X5) [obs] at (5,7) {\scriptsize $x_5$};
  \node<1-4> (X6) [obs] at (10,2) {\scriptsize $x_6$};
  \node<1-4> (X7) [obs] at (13,3) {\scriptsize $x_7$};
  \node<1-4> (X8) [obs] at (11,1) {\scriptsize $x_8$};
  \node<1-4> (X9) [obs] at (1,2) {\scriptsize $x_9$};
  \node<1-4> (X1) [obs] at (5,1) {\scriptsize $x_{1}$};
  
  \node<3> (Ti) [post, above right] at (Xi) {\scriptsize $\boldsymbol\tau_{i} = (\tau_{i1}, \tau_{i2}, \tau_{i3}, \tau_{i4}$)};
  
  \node<4-> (Ti) [post] at (7,4) {\scriptsize $\boldsymbol\tau_i$};
  \node<5-> (T2) [post] at (2,2.5) {\scriptsize $\boldsymbol\tau_2$};
  \node<5-> (T3) [post] at (4,2) {\scriptsize $\boldsymbol\tau_3$};
  \node<5-> (T4) [post] at (1,4) {\scriptsize $\boldsymbol\tau_4$};
  \node<5-> (T5) [post] at (5,7) {\scriptsize $\boldsymbol\tau_5$};
  \node<5-> (T6) [post] at (10,2) {\scriptsize $\boldsymbol\tau_6$};
  \node<5-> (T7) [post] at (13,3) {\scriptsize $\boldsymbol\tau_7$};
  \node<5-> (T8) [post] at (11,1) {\scriptsize $\boldsymbol\tau_8$};
  \node<5-> (T9) [post] at (1,2) {\scriptsize $\boldsymbol\tau_9$};
  \node<5-> (T1) [post] at (5,1) {\scriptsize $\boldsymbol\tau_{1}$};
  
%   \node (T2) [post, above right=3] at (X2) {\scriptsize $\boldsymbol\tau_{2}$};
%   \node (T3) [post, above right=3] at (X3) {\scriptsize $\boldsymbol\tau_{3}$};
%   \node (T4) [post, above left=3] at (X4) {\scriptsize $\boldsymbol\tau_{4}$};
%   \node (T5) [post, above right=3] at (X5) {\scriptsize $\boldsymbol\tau_{5}$};
%   \node (T6) [post, above left=3] at (X6) {\scriptsize $\boldsymbol\tau_{6}$};
%   \node (T7) [post, above right=3] at (X7) {\scriptsize $\boldsymbol\tau_{7}$};
%   \node (T8) [post, below right=3] at (X8) {\scriptsize $\boldsymbol\tau_{8}$};
%   \node (T9) [post, above left=3] at (X9) {\scriptsize $\boldsymbol\tau_{9}$};
%   \node (T10) [post, below right=3] at (X10) {\scriptsize $\boldsymbol\tau_{10}$};
  %\node (C2) [class] at(\origin-\osep, \lobs) {$C_1$};
  %\node (C3) [class] at(\origin-\osep, \lobs) {$C_1$};
%     \draw (0,0) [draw=none] grid (2*\origin,8);
%     
%     \node (X1) [obs] at (\origin-\osep,\lobs) {$x_1$};
%     \node (C1) [class] at(\origin-\osep, \lobs) {$C_1$};
%     \node (Xi) [obs] at (\origin,\lobs) {$x_i$};
%     \node (Xn) [obs] at (\origin+\osep,\lobs) {$x_n$};
% 
%     \node (T11) [post] at (\origin-\psep-\osep, \lpost) {$\tau_{11}$};
%     \node (T1j)  [post] at (\origin - \osep,  \lpost) {$\tau_{1 j}$};
%     \node (T1k) [post] at (\origin+\psep-\osep, \lpost) {$\tau_{1k}$};
% 
%     \node  (Ti1) [post] at (\origin-\psep,   \lpost) {$\tau_{i1}$};
%     \node  (Tij)  [post] at (\origin, \lpost) {$\tau_{i j}$};
%     \node  (Tik) [post] at (\origin+\psep, \lpost) {$\tau_{ik}$};
% 
%     \node (Tn1) [post] at (\origin-\psep+\osep,\lpost) {$\tau_{n1}$};
%     \node (Tnj)  [post] at (\origin + \osep, \lpost) {$\tau_{n j}$};
%     \node (Tnk) [post] at (\origin+\psep+\osep, \lpost) {$\tau_{nk}$};
% 
%     \path (X1) -- (Xi) node [font=\Huge, midway, sloped] {$\dots$};
%     \path (Xi) -- (Xn) node [font=\Huge, midway, sloped] {$\dots$};
% 
%     \path[dotted] (X1) edge (T11);
%     \path[dotted] (X1) edge (T1j);
%     \path[dotted] (X1) edge (T1k);
% 
%     \path[dotted] (Xi) edge (Ti1);
%     \path[dotted] (Xi) edge (Tij);
%     \path[dotted] (Xi) edge (Tik);
% 
%     \path[dotted] (Xn) edge (Tn1);
%     \path[dotted] (Xn) edge (Tnj);
%     \path[dotted] (Xn) edge (Tnk);
\end{tikzpicture}