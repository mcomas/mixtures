%\documentclass[preprint, review, 3p, authoryear]{elsarticle}
\documentclass[10pt, a4paper]{article}

\usepackage{setspace}
\usepackage[utf8]{inputenc}
\usepackage{amsmath, amssymb, amsthm}
\usepackage{xcolor}
\usepackage{graphicx}
\usepackage[authoryear]{natbib}
\usepackage{apalike}

\DeclareMathOperator*{\argmax}{arg\,max}

\newtheorem{prob}{Problem}
\newtheorem{prop}{Proposition}
\newtheorem{definition}{Definition}

\title{From merging Gaussian mixtures to merging generic classes}
\author{M. Comas-Cufí \and J.A. Martín-Fernández \and G. Mateu-Figueras}

\doublespacing
\begin{document}

\maketitle

\section{Introduction}

Different he problem of merging the components of a Gaussian mixtures has received special atention \cite{melnykov2013distribution,lee2004combining,hennig2010methods,baudry2010combining,pastore2013merging}. 

Hennig summarizes the hierarchical principle for merging Gaussian components as follows:
\begin{enumerate}
\item Start with all components of the initially estimated Gaussian mixture as current clusters
\item Find the pair of current clusters most promising to merge
\item Apply a stopping criterion to decide wheter to merge them to form a new current cluster, or to use the current clustering as the final one.
\item If merged, go to 2.
\end{enumerate}


In this paper we focus on methods based on misclassification probabilities. Concretely, on those depending on the posterior probabilities \citep{melnykov2013distribution}, \citep{baudry2010combining} and \citep[in \textsc{demp} approach]{hennig2010methods}.


We exposes the problem in a general setting which contains the approaches proposed by Baudry and Hennig.

\section{The subjectiveness of clustering decisions}

It is well known that there is a strong subjective component in the decision of what a ``true cluster'' is \citep{hennig2010methods}.

\begin{prob}
Given a compositional sample $T = \{ \boldsymbol{\tau_1}, \dots, \boldsymbol{\tau_n} \}$, with $\boldsymbol{\tau_i} = (\tau_{i1}, \dots, \tau_{iK})$ denoting the pertinence to classes $C_1, \dots, C_K$, build a hierarchy over the set of classes.
\end{prob}


\bibliographystyle{apalike}
\bibliography{tex/combining_mixtures}{}

\end{document}