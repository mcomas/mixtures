%\documentclass[preprint, review, 3p, authoryear]{elsarticle}
\documentclass[10pt, a4paper]{article}

%\usepackage{setspace}
\usepackage[utf8]{inputenc}
\usepackage{amsmath, amssymb, amsthm, bbm}
\usepackage{xcolor}
\usepackage{graphicx}
\usepackage[authoryear]{natbib}
\usepackage{apalike}
\usepackage{relsize}
\usepackage{array}
\usepackage{multirow}

\DeclareMathOperator*{\argmax}{arg\,max}

\newtheorem{prob}{Problem}
\newtheorem{prop}{Proposition}
\newtheorem{definition}{Definition}

%%%%% bold symbol in math enviornment
\newcommand{\m}[1]{\boldsymbol{#1}}

\title{Merging the components of a finite mixture using  posterior probabilities}
\author{M. Comas-Cufí \and J.A. Martín-Fernández \and G. Mateu-Figueras}

%\doublespacing
\begin{document}

\maketitle

\section{Introduction}

%Cl
The most standard parametric approach in cluster analysis assumes data can be modelled by a finite mixture distribution. The approach has two steps; first, a finite mixture distribution with probability density function
\[
f(\;\cdot\; ; \pi_1, \dots, \pi_k, \m\theta_1 \dots \m\theta_k) = \pi_1   (\;\cdot\; ; \m\theta_1) + \dots + \pi_k f(\;\cdot\; ; \m\theta_k),
\]%
%with $\sum_{j=1}^k = \pi_j =1$
is fitted to a sample $\m X$, obtaining estimates $\hat{\pi}_1, \dots, \hat{\pi}_k$ and $\hat{\m\theta}_1 \dots \hat{\m\theta}_k$. After the fitting process, each observation $\m x$ is assigned to the finite mixture components $j$, $1\leq j \leq k$, with $\hat{\pi}_j f(\m x ; \hat{\m\theta}_j)$ maximum. 

In the previous approach, it is common to have mixture components not separated enough from other mixture components, and therefore, it is difficult to consider them as two different cluster. To deal with this situation, different authors propose to consider two or more components to form a unique cluster, and therefore, to combine or to merge different components into one single component ({\color{red} Cites rellevants per el problema d'ajuntar components}).

In \cite{hennig2010methods} the author separates the existing approaches in two different groups: those methods based in modality and those methods based on misclassification probabilities. 

In this paper we deal with methods that are based on the posterior probabilities, $\hat{\tau}_{ij}$, that an observation $\m x_i$, $1\leq i \leq n$, is generated by component $j$, $1\leq j\leq k$ \citep{longford2014,melnykov2013distribution,hennig2010methods,baudry2010combining}. All this methods can be included inside the category of methods based on misclassification probabilities. Although this approaches have been presented for Gaussian mixtures, their nature allows them to be applied using other probability distribution mixtures.


% , $\hat{\tau}_{ij}$, obtained after fitting the finite mixture model to our data. $\hat{\tau}_{ij}$ is defined to be the posterior probability of observation $i$, $1 \leq i \leq n$ being generated by component $j$, $1\leq j\leq k$ after adjusting a mixture with $k$ components  to a sample $X$ with $n$ observations \citep{longford2014,melnykov2013distribution,hennig2010methods,baudry2010combining}.  Although this approaches have been presented for Gaussian mixtures, their nature allows them to be applied in any kind of probability distribution mixture.

The paper is organised as follows: In Section~\ref{definitions} we introduce two concepts; \emph{the partition of a finite mixture distribution} and \emph{the hierarchical combination of components}. The definitions are going to be useful to standardise notation between different methods. We show a simple example to get used to the new notation. In Section~\ref{old_methods}, the approaches to obtain a hierarchical combination of components appeared in \cite{hennig2010methods} and \cite{baudry2010combining} are presented using the new notation. 

%Moreover, a new method based on log-ratios is introduced. In Section~\ref{confusion} the problems a reformulated in a common approach, they are compared to the approach presented in \citep{longford2014} 


%In general, \cite{hennig2010methods} summarises the algorithm of hierarchically merging Gaussian components as follows:
%\begin{enumerate}
%\item Start with all components of the initially estimated Gaussian mixture as current clusters
%\item Find a pair of components to merge and forming a single cluster
%%\item Calculate the posterior probability of pertinence to a cluster 
%\item Apply a stopping criterion to decide whether to merge them to form a new current cluster, or to use the current clustering as the final one.
%\item If merged, go to 2.
%\end{enumerate}

%As commented, in this paper we only focus on methods based on the posterior probabilities. Our aim is to find strategies to hierachically merge components into clusters. Let us remark that we will not focus on stopping criteria, and therefore we will hierarchically merge all components until a single cluster with all components is obtained.




% \section{The subjectiveness of clustering decisions}
% 
% It is well known that there is a strong subjective component in the decision of what a ``true cluster'' is \citep{hennig2010methods}.
%
% \begin{prob}
% Given a compositional sample $T = \{ \boldsymbol{\tau_1}, \dots, \boldsymbol{\tau_n} \}$, with $\boldsymbol{\tau_i} = (\tau_{i1}, \dots, \tau_{iK})$ denoting the pertinence to classes $C_1, \dots, C_K$, build a hierarchy over the set of classes.
% \end{prob}

\section{Definitions}
\label{definitions}

Let $\mathbb{X}$ be a sample space. A \emph{finite mixture of distributions} is a probability distribution with pdf defined as the linear combination of pdf from other probability distributions all defined in $\mathbb{X}$. In general, the pdf $f$ of a finite mixture of distributions is
\begin{equation}\label{mixt}
f(\;\cdot\; ; \pi_1, \dots, \pi_k, \m\theta_1 \dots \m\theta_k) = \pi_1 f_1(\;\cdot\; ; \m\theta_1) + \dots + \pi_k f_k(\;\cdot\; ; \m\theta_k),
\end{equation}
where $\m\theta_1, \dots,  \m\theta_k$ are the parameters of the pdf $f_1, \dots, f_k$ respectively and, because $f$ is a pdf, we have the restriction $\sum_{\ell = 1}^k \pi_\ell = 1$. The probability distributions $f_j$, $1 \leq j \leq k$, are called the \emph{components} of the finite mixture $f$, or \emph{mixture components}.

Let $f$ be a finite mixture of distributions with  parameters  $\pi_1, \dots, \pi_k, \m\theta_1 \dots \m\theta_k$ as defined in Equation~\ref{mixt}, and let $I$  be a subset of $\{1, \dots, k\}$. We denote by $f_I$ the finite mixture of distributions with pdf defined by
\[
f_I = \sum_{j \in I} \frac{\pi_i}{\pi_I} f_j(\;\cdot\; ; \m\theta_j)
\]
where $\pi_I = \sum_{\ell \in I} \pi_\ell$. To simplify, we do not specify the parameters of $f_I$, which are parameters borrowed from $f$. Note that using this notation, we have that $f_{\{1, \dots, k\}} = f$ and $f_{\{j\}} = f_j$.

A \emph{partition} $\mathcal{I}$ of $\{1, \dots, k\}$ is a set of subsets of $\{1, \dots, k\}$, called $parts$, such that $\bigcup_{I \in \mathcal{I}} I = \{1, \dots, k\}$ and  if two parts $I$, $J$ are different, $I \cap J = \emptyset$ holds. To simplify, through this paper we assume an order within the elements of a partition. Doing so, we can index the partition and write $\mathcal{I} = \{ I_1, \dots, I_s\}$. Importantly, given any partition $\mathcal{I} = \{ I_1, \dots, I_s\}$, the mixture $f$ (Eq.~\ref{mixt}) can be rewritten as:
\[
f = \pi_{I_1} f_{I_1} + \dots + \pi_{I_s} f_{I_s}.
\]


A \emph{hierarchical combination of components} is a sequence of partitions $\mathcal{I}_1, \dots, \mathcal{I}_k$ of $\{1,...,k\}$, where $\mathcal{I}_1$ is the one-part partition $\mathcal{I}_1 = \{ \{1, \dots, k\} \}$, and for each $k'$, $1 <  k' \leq k$,
\begin{itemize}
\item $\mathcal{I}_{k'}$ has $k'$ elements  and
\item if a part $I \in \mathcal{I}_{k'-1}$ then either there is a part $J \in \mathcal{I}_{k'}$ with $J = I$ or there are two parts $J_1, J_2 \in \mathcal{I}_i$ with $I = J_1 \cup J_2$.
\end{itemize}


\subsection*{Model-based clustering}

When model-based clustering is based on finite mixtures, a common approach is to assume that a sample is following a finite mixture of distributions $f_1, \dots, f_k$, and then proceed as follows, 
\begin{enumerate}
\item finding a suitable estimators $\hat{\pi}_1, \dots, \hat{\pi}_k,$ $\hat{\m\theta}_1, \dots, \hat{\m\theta}_k$ of parameters $\pi_1, \dots, \pi_k,$ $\m\theta_1, \dots, \m\theta_k$, and
\item classifying each observation according to the maximum a posteriori criteria, i.e., two observations $\m x, \m y \in \mathbb{X}$ are classified to same cluster if and only if
\[
\argmax_{j=1}^k \frac{ \hat{\pi}_j f_j(\m x ; \hat{\m\theta}_j) }{\sum_{\ell=1}^k \hat{\pi}_\ell f_\ell(\m x ; \hat{\m\theta}_\ell) } = \argmax_{j=1}^k \frac{ \hat{\pi}_j f_j(\m y ; \hat{\m\theta}_j) }{ \sum_{\ell=1}^k \hat{\pi}_\ell f_\ell(\m y ; \hat{\m\theta}_\ell) }.
\]
\end{enumerate}


\cite{lee2004combining,hennig2010methods,baudry2010combining,melnykov2013distribution,pastore2013merging} noted that classifying an element according to the probability of belonging to one component could be misleading. Instead, they propose that one cluster can be formed by the combination of different mixture components. Formally, given a partition $\mathcal{I} = \{ I_1, \dots, I_s\}$, two elements $\m x, \m y \in \mathbb{X}$ are classified to the same cluster if and only if
\begin{equation}\label{cluster_criteria}
\argmax_{j=1}^s \frac{ \hat{\pi}_{I_j} \hat{f}_{I_j}(\m x) }{\sum_{\ell=1}^s \hat{\pi}_{I_\ell} \hat{f}_{I_\ell}(\m x ) } = \argmax_{j=1}^s \frac{ \hat{\pi}_{I_j} \hat{f}_{I_j}(\m y) }{ \sum_{\ell=1}^s \hat{\pi}_{I_\ell} \hat{f}_{I_\ell}(\m y) }
\end{equation}

where $\hat{f}_{I_j}(\; \cdot \;) = \sum_{j' \in I_j} \frac{\hat{\pi}_{j'}}{\hat{\pi}_{I_j}} f_{j'}(\; \cdot \; ; \hat{\m\theta}_{j'})$ and $\hat{\pi}_{I_j} =  \sum_{j' \in I_j} \hat{\pi}_{j'}$. 

Let $X = \{\m x_1\dots, \m x_n\}$ be a sample defined in $\mathbb{X}$. Along this paper, given a partition $\mathcal{I} = \{ I_1, \dots, I_s \}$ we define the posterior probability  of $\m x_i$ being classified to $I_j$ as
\[
\hat{\tau}_{i I_j} =  \frac{ \hat{\pi}_{I_j} \hat{f}_{I_j}(\m x_i) }{\sum_{\ell=1}^s \hat{\pi}_{I_\ell} \hat{f}_{I_\ell}(\m x_i)},
\]
and, for a partition  $\mathcal{I} = \{ I_1, \dots, I_s\}$, we define the posterior probability vector
\[
\hat{\tau}_{i \mathcal{I}} = \left( \hat{\tau}_{i I_1} , \dots, \hat{\tau}_{i I_s}  \right).
\]
Note that whenever  $\mathcal{I} = \{ I_1, \dots, I_s\}$ is a partition, $\sum_{j=1}^s \hat{\tau}_{i I_j} = 1$ for $1 \leq i \leq n$.

For the sake of an example consider the following estimated Gaussian mixture

\[
\hat{f} = \sum_{j=1}^6 \hat{\pi}_j \phi(\;\cdot\; ; \hat{\m\mu}_j, \hat{\m\Sigma}_j)
\]
with parameters
{\small
\[
\begin{array}{l@{\hskip 0.1in}l@{\hskip 0.1in}c }
\hat{\pi}_1 = 0.13, & \hat{\m\mu}_1 = \left(10.8,69.17\right), & \hat{\m\Sigma}_1 = \left(
\begin{array}{cc}
36.41&1.45 \\ 
1.45&55.13 \\ 
\end{array}
\right), \\ & &\\ 
\end{array}
\]
\[
\begin{array}{l@{\hskip 0.1in}l@{\hskip 0.1in}c }
\hat{\pi}_2 = 0.09, & \hat{\m\mu}_2 = \left(32.68,22.46\right), & \hat{\m\Sigma}_2 = \left(
\begin{array}{cc}
26.76&1.07 \\ 
1.07&40.52 \\ 
\end{array}
\right), \\ & &\\ 
\end{array}
\]
\[
\begin{array}{l@{\hskip 0.1in}l@{\hskip 0.1in}c }
\hat{\pi}_3 = 0.07, & \hat{\m\mu}_3 = \left(13.65,51.91\right), & \hat{\m\Sigma}_3 = \left(
\begin{array}{cc}
33.95&1.35 \\ 
1.35&51.39 \\ 
\end{array}
\right), \\ & &\\ 
\end{array}
\]
\[
\begin{array}{l@{\hskip 0.1in}l@{\hskip 0.1in}c }
\hat{\pi}_4 = 0.16, & \hat{\m\mu}_4 = \left(83.8,4.21\right), & \hat{\m\Sigma}_4 = \left(
\begin{array}{cc}
82.27&3.28 \\ 
3.28&124.56 \\ 
\end{array}
\right), \\ & &\\ 
\end{array}
\]
\[
\begin{array}{l@{\hskip 0.1in}l@{\hskip 0.1in}c }
\hat{\pi}_5 = 0.24, & \hat{\m\mu}_5 = \left(41.28,19.51\right), & \hat{\m\Sigma}_5 = \left(
\begin{array}{cc}
55.87&2.23 \\ 
2.23&84.59 \\ 
\end{array}
\right) \text{ and} \\ & &\\ 
\end{array}
\]
\[
\begin{array}{l@{\hskip 0.1in}l@{\hskip 0.1in}c }
\hat{\pi}_6 = 0.32, & \hat{\m\mu}_6 = \left(24.69,66.04\right), & \hat{\m\Sigma}_6 = \left(
\begin{array}{cc}
57.85&2.3 \\ 
2.3&87.58 \\ 
\end{array}
\right). \\ & &\\ 
\end{array}
\]

}

The mixture $\hat{f}$ with $6$ components has been obtained as follows:
\begin{enumerate}
\item A sample  $X_{500}=\{\m x_1, \dots, \m x_{500}\}$ has been generated from a Gaussian mixture \emph{with 3 components}. The generation was done using the R package \textsc{MixSim}. The overlapping between components $\omega$ is a measure that defined the overlapping between components in a mixture (see Melnykov for more details). To generate sample $X_{500}$ a maximum overlapping of  $\check{\omega} = 0.01$ was fixed.
\item A Gaussian mixture \emph{with 6 components} has been fitted to sample $X_{500}$. To fit the finite mixture the R package \textsc{Rmixmod} was used.
\end{enumerate}
In Figure~\ref{ex_mixture} the sample is represented with the isodensity curves of the adjusted mixture $\hat{f}$. The estimated parameter $\hat{\mu}_j$ of each component is represented by a cross.

\begin{figure}[thbp]
\begin{center}
\begin{tabular}{cc}
 %   6 toy mixture
  \includegraphics[trim=0cm 0cm 0cm 0cm,width=0.6\textwidth]{figures/partition-example-mixture.pdf} \\
 \end{tabular}
 \caption{Density of a Gaussian mixture with 6 component adjusted to a data set generated from a 3 component Gaussian mixture using R \textsc{MixSim} package with max overlapping of $\check{\omega} = 0.01$. Sample mean of each component is represented by '+'.}\label{ex_mixture}
\end{center}
\end{figure}

Let $I = \{1,2,3,4,5,6\}$. As commented before, any partition of $I$ yields to a feasible final classification using Eq.\ref{cluster_criteria}. For example, consider the partition 
\[\mathcal{I}_6 = \{\{1\},\{2\},\{3\},\{4\},\{5\},\{6\}\}\]
of $I$. Using partition $\mathcal{I}_6$ every observation $\m x_i \in \mathbb{R}^2$ was assigned to the part $\{j\}$ with maximum $\hat{\tau}_{i\{j\}}$. In Figure~\ref{ex_part6} each observation $\m x_i$ has been separated according to the assigned part. Moreover, the isodensity curves for each of the pdf $\hat{f}_{\{j\}} = \phi(\;\cdot\; ; \hat{\m\mu}_j, \hat{\m\Sigma}_j)$, $1\leq j \leq 6$, have been included in the graphic. 

%In other words, $\m x_i \in \mathbb{R}^2$ is assigned to cluster labeled
%\[
%\argmax_{j=1}^6 \frac{ \hat{\pi}_{\{j\}} \hat{f}_{\{j\}}(\m x) }{\sum_{\ell=1}^6 \hat{\pi}_{I_\ell} \hat{f}_{I_\ell}(\m x ) }
%\]


%$\hat{f}_{\{j\}}(\;\cdot\;) = \frac{\hat{\pi}_j}{\hat{\pi}_j} f(\;\cdot\;; \hat{\m\mu}_j, \hat{\m\Sigma}_j) = f(\;\cdot\;; \hat{\m\mu}_j, \hat{\m\Sigma}_j)$ for $j = 1,...,6$. Partition $\mathcal{I}_6$ yields to the cluster algorithm where an element $x_i$ is assigned to the part $\{j\}$ of $\mathcal{I}_6$ such that $\hat{\tau}_{i\{j\}}$ is maximum.


\begin{figure}[!h]
\begin{center}
\begin{tabular}{cc}
 %   6 toy mixture
  \includegraphics[trim=0cm 0cm 0cm 0cm,width=\textwidth]{figures/partition-example-part6.pdf} \\
 \end{tabular}
 \caption{ Initial classifIation with 6 clusters: each observation was assigned to a single component.}\label{ex_part6}
\end{center}
\end{figure}

In contrast, if we consider the partition
\[\mathcal{I}_3 = \{\{1, 3, 6\},\{2, 5\},\{4\}\}\]
which is grouping those components with closest mean, we get the 3 clusters given in Figure~\ref{ex_part3a}. We have included the isodensity curves of pdf $\hat{f}_{\{1,3,6\}}$, $\hat{f}_{\{2, 5\}}$ and $\hat{f}_{\{4\}}$ given by

\[ 
%\left\{ 
\begin{array}{r c l}
\hat{f}_{\{1,3,6\}} & = & \frac{1}{0.52}(0.13 \phi(\;\cdot\; ; \hat{\m\mu}_1, \hat{\m\Sigma}_1) + 0.07 \phi(\;\cdot\; ; \hat{\m\mu}_3, \hat{\m\Sigma}_3) + 0.32 \phi(\;\cdot\; ; \hat{\m\mu}_6, \hat{\m\Sigma}_6)), \\
\hat{f}_{\{2, 5\}} & = &  \frac{1}{0.33}(0.09 \phi(\;\cdot\; ; \hat{\m\mu}_2, \hat{\m\Sigma}_2) + 0.24 \phi(\;\cdot\; ; \hat{\m\mu}_5, \hat{\m\Sigma}_5)), \\
\hat{f}_{\{4\}} & = &\phi(\;\cdot\; ; \hat{\m\mu}_4, \hat{\m\Sigma}_4).
\end{array} 
%\right. 
\]
  


\begin{figure}[!h]
\begin{center}
\begin{tabular}{cc}
  \includegraphics[trim=0cm 0cm 0cm 0cm,width=0.6\textwidth]{figures/partition-example-part3a.pdf} \\
 \end{tabular}
 \caption{Classifying assigning each observation to a single component}\label{ex_part3a}
\end{center}
\end{figure}

%Finally, suppose that instead of grouping by similar mean, we decide to group by similar covariance matrix. By choosing partition
%\[\mathcal{I}'_3 = \{\{1, 2, 3\},\{4\},\{5, 6\}\},\]
%we end up with the clustering given in Figure~\ref{ex_part3b}.
%
%\begin{figure}[!h]
%\begin{center}
%\begin{tabular}{cc}
%  \includegraphics[trim=0cm 0cm 0cm 0cm,width=0.6\textwidth]{figures/partition-example-part3b.pdf} \\
% \end{tabular}
% \caption{Classifying assigning each observation to a single component}\label{ex_part3b}
%\end{center}
%\end{figure}

Consider the following hierarchical combination of components given by the following sequence of partitions
\begin{equation}
\begin{array}{r c c}
\mathcal{H}(\mathcal{I}) &=& \{ \{\{1\},\{2\},\{3\},\{4\},\{5\},\{6\}\}, \\
   & & \{\{1, 6\},\{2\},\{3\},\{4\},\{5\} \}, \\
   & &    \{\{1, 6, 3\},\{2\},\{4\},\{5\} \}, \\
   & &    \{\{1, 6, 3\},\{2, 5\},\{4 \} \}, \\
    & &   \{\{1, 6, 3\},\{2, 4, 5\} \}, \\
   & &    \{\{1, 2, 3, 4, 5, 6\}\} \}.
\end{array}
\label{hier_ex}
\end{equation}
Each partition belonging to the hierarchical combination of components given by \ref{hier_ex} defines a clustering where each element is classified to one of the parts of each partition. Therefore, a hierarchical combination of components defines a hierarchical clustering structure. Figure~\ref{hierarchical} show that each partition defines a clustering on sample $X$. The first partition $\{\{1\},\{2\},\{3\},\{4\},\{5\},\{6\}\}$ defines a clustering with 6 clusters, the partition $\{\{1, 6\}, \{3\},\{2\},\{4\},\{5\} \}$ defines a clustering with 5 clusters, and so on.

\begin{figure}[thbp]
\begin{center}
\begin{tabular}{cc}
  \includegraphics[trim=0cm 0cm 0cm 0cm,width=\textwidth]{figures/partition-example-part6.pdf} \\
    \includegraphics[trim=0cm 0cm 0cm 0cm,width=0.83\textwidth]{figures/partition-example-part5.pdf} \\
      \includegraphics[trim=0cm 0cm 0cm 0cm,width=0.67\textwidth]{figures/partition-example-part4.pdf} \\
        \includegraphics[trim=0cm 0cm 0cm 0cm,width=0.5\textwidth]{figures/partition-example-part3a.pdf} \\
          \includegraphics[trim=0cm 0cm 0cm 0cm,width=0.33\textwidth]{figures/partition-example-part2.pdf} \\
            \includegraphics[trim=0cm 0cm 0cm 0cm,width=0.2\textwidth]{figures/partition-example-part1.pdf}
 \end{tabular}
 \caption{Hierarchical cluster obtained by the hierarchical combination of components.}\label{hierarchical}
\end{center}
\end{figure}

In this article, we are not interested in how to decide which of the different partitions in a hierarchical combination of components defines a ``better'' clustering. Instead, we propose different methods to decide how to merge the components. More concretely, given a finite mixture we are interested in analysing methods to decide how to merge components only using information given by the posterior probabilities $\hat{\tau}_{i \mathcal{I}}$, $1\leq i \leq n$. {\color{red}No m'agrada aquest paràgraf, no em sona bé començar que no estem interessats amb algo}


\section{Hierarchical algorithms based on posterior probabilities}
\label{old_methods}

%In this section we review different approaches to build a hierarchical combination of components using the information contained in the posterior probabilities $\hat{\tau}_{iI}$. The motivation of this section are: to reformulate the different approaches with the same notation, and to indicate the idea behind this approaches. Moreover, in this section we present a new method to generate a hierarchical combination of components. The new approach is based the log-ratios.

Once a finite mixture distribution is adjusted to a given sample $X=\{x_1, \dots, x_n\}$, for a partition $\mathcal{I}_s = \{ I_1, \dots, I_s\}$ we can calculate the posterior probability vector   $\hat{\m\tau}_{i \mathcal{I}_s} = \left( \hat{\tau}_{i I_1} , \dots, \hat{\tau}_{i I_s}  \right)$ for $i$, $1 \leq i \leq n$.

In this section we show different approaches to combine two parts from partition $\mathcal{I}_s$, $I_a$ and $I_b$, into one part $I_a \cup I_b$. The shown methods only use the information contained in the posterior probability vectors $\{ \hat{\m\tau}_{1 \mathcal{I}_s},\dots, \hat{\m\tau}_{n \mathcal{I}_s} \}$.


\subsection*{Algorithm based on the total Entropy}

%Let ${\boldsymbol\tau}_1, \dots, {\boldsymbol\tau}_n$ be the probability vectors giving the probability that elements $\textbf{x}_1, \dots, \textbf{x}_n$ belongs to classes $C_1, \dots, C_k$.  

%Once a finite mixture distribution is adjusted to a sample $X=\{x_1, \dots, x_n\}$. Given a partition $\mathcal{I}_s = \{ I_1, \dots, I_s\}$, for each element $\m x_i$ we can calculate the posterior probabilities of $\m x_i$ being classified to each part, obtaining \emph{the posterior probability composition} $\hat{\tau}_{i \mathcal{I}_s} = \left( \hat{\tau}_{i I_1} , \dots, \hat{\tau}_{i I_s}  \right)$ summing up to one. \marginpar{Potser això estaria millor a definicions}

Let $\hat{\m \tau}_{i \mathcal{I}_{s, a \cup b}}$ be the posterior probability vector obtained  after joining the part $I_a$ and $I_b$ from $ \mathcal{I}_s$ into one part $I_{a \cup b}$. 

The Entropy of a posterior probability vector $\hat{\m \tau}_{i \mathcal{I}_s}$ is
\[
Ent( \hat{\m \tau}_{i \mathcal{I}_s} ) = \sum_{j=1}^s \hat{\tau}_{i I_j}  log(\hat{\tau}_{i I_j} ).
\]
$Ent( \hat{\m \tau}_{i \mathcal{I}_s} )$ is a convex function with its minimum at $(\frac{1}{s},\dots,\frac{1}{s})$. In terms of posterior probabilities, the minimum is the point where the posterior probability of an observation $x_i$ to be generated from distributions $\hat{f}_{I_1}, \dots, \hat{f}_{I_s}$ is the same.

 \cite{baudry2010combining} proposes to merge the components, $I_a$ and $I_b$, such that $\sum_{i=1}^n Ent( \hat{\m \tau}_{i \mathcal{I}_{s, a \cup b}} )$ is minimum. The criteria is equivalent to combine the parts, $I_a$ and $I_b$, such that the difference 
\begin{multline*}
\Delta Ent(\hat{\m \tau}_{i \mathcal{I}_s}, a, b) = \sum_{i=1}^n Ent( \hat{\m \tau}_{i \mathcal{I}_{s, a \cup b}}) - \sum_{i=1}^n Ent( \hat{\m \tau}_{i \mathcal{I}_s}) =  \\ = \sum_{i=1}^n  \left( \hat{\tau}_{i I_{a\cup b}}  log(\hat{\tau}_{i I_{a\cup b}} ) +  \sum_{\substack{j=1 \\
                                                            j \neq a, b}}^s \hat{\tau}_{i I_j}  log(\hat{\tau}_{i I_j} ) \right)  - \sum_{i=1}^n \sum_{j=1}^s \hat{\tau}_{i I_j}  log(\hat{\tau}_{i I_j} ) = \\  =   \sum_{i=1}^n  (\hat{\tau}_{iI_a}+\hat{\tau}_{iI_b}) \log(\hat{\tau}_{iI_a} + \hat{\tau}_{iI_b}) - \sum_{i=1}^n \left\{ \hat{\tau}_{iI_a} \log(\hat{\tau}_{iI_a}) + \hat{\tau}_{iI_b} \log(\hat{\tau}_{iI_b})\right\}
\end{multline*}
is maximum.

It is noteworthy to remark that function $\Delta Ent(\hat{\m \tau}_{i \mathcal{I}_s}, a, b)$ is symmetric, i.e. $\Delta Ent(\hat{\m \tau}_{i \mathcal{I}_s}, a, b) = \Delta Ent(\hat{\m \tau}_{i \mathcal{I}_s}, b, a)$.

Consider the finite mixture
\begin{equation}\label{two_mixture}
f = \pi_a \hat{f}_{I_a} + (1 - \pi_a) \hat{f}_{I_b, \lambda}
\end{equation}
where $\hat{f}_{I_a} = N(0, 1)$ is the normal with mean equal to $0$ and variance $1$, $\hat{f}_{I_b, \lambda} = N(\lambda, 1)$ to be the normal with mean equal to $\lambda$ and variance $1$ and $\pi_a = 0.1, 0.2, 0.3, 0.4, \dots, 0.9$. 

We have approximated the expected value of the total entropy
\[
\mathbb{E}\left[ \Delta Ent\left(
 \left( 
 \frac{\pi_a \hat{f}_{I_a}(\m x)}{\pi_a \hat{f}_{I_a}(\m x) + (1-\pi_a) \hat{f}_{I_b, \lambda}(\m x)}, 
 \frac{(1-\pi_a) \hat{f}_{I_b, \lambda}(\m x)}{\pi_a \hat{f}_{I_a}(\m x) + (1-\pi_a) \hat{f}_{I_b, \lambda}(\m x)} \right), a, b\right) | \m x \sim f \right]
\]
for different value of $\lambda$ and $\pi_a$
We have generated $100000$ observations following mixture given by Equation~\ref{two_mixture} 
%(we have generated $\pi_a * n$ values from $hat{f}_{I_a}$ and $(1-\pi_a) * n$ values from $hat{f}_{I_b}$), 
and we have calculate the total Entropy (the mean) for this sample.

In Figure~\ref{fig:mu_varying} (top-left) the total Entropy is represented for $\lambda$, $0 \leq \lambda \leq 3$ and for $\pi_a \in \{ 0.1, 0.2, 0.3, 0.4, 0.5, 0.6, 0.7, 0.8, 0.9\}$. In the plot, because the total entropy is symmetric respect $a$ and $b$, the curves for $\pi_a = 1-\pi_a$ overlap. We can see that for a fixed $\lambda$, the higher confusion appears when $\pi_a = 0.5$.

\begin{figure}[!t]
\centering
\includegraphics[scale=.7]{figure01/fig01.eps}
\caption{$\lambda$ varying}
\label{fig:mu_varying}
\end{figure}

%To ilustrate the total Entropy method consider the following scenario: suppose $\hat{f}_{I_a} = N(0, 1)$ is the standardized normal distribution and let $\hat{f_\lambda}_{I_b} = N(\lambda, 1)$. In Figure~\ref{entropy_plot} the value

%\lambda_{\text{Ent}}(\hat{\tau}_{i \mathcal{I}_s}, I_a, I_b) := 

\subsection*{DEMP algorithm}

DEMP method was introduced in \cite{hennig2010methods}. The approach proposes to combine the two parts $I_a$ and $I_b$ from $ \mathcal{I}_s$ such that \emph{the probability of classifying an observation generated from component $\hat{f}_{I_a}$ to component $\hat{f}_{I_b}$} is maximum.

To estimate the previous probability,  \cite{hennig2010methods} suggests to use a consistent estimator (DEMP). The estimator takes the form
\[
\frac{ \frac{1}{n} \sum_{i=1}^n {\hat{\tau}_{iI_a} \mathbbm{1}\left( \forall j\; \hat{\tau}_{i I_{b}} \geq \hat{\tau}_{iI_j} \right)}}{ \hat{\pi}_{I_a}},
\]
where $\mathbbm{1}\left( \cdot \right)$ is the indicator function. Because $ \hat{\pi}_{I_a} = \frac{1}{n} \sum_{i=1}^n \hat{\tau}_{iI_a}$, the estimator can be rewritten in terms of the posterior probability vector as
\begin{equation}\label{demp_criteria}
DEMP(\hat{\m \tau}_{i \mathcal{I}_s}, a, b) =\frac{ \sum_{i=1}^n {\hat{\tau}_{iI_a} \mathbbm{1}\left( \forall j\; \hat{\tau}_{i I_{b}} \geq \hat{\tau}_{iI_j} \right)}}{\sum_{i=1}^n \hat{\tau}_{iI_a} }.
\end{equation}

In contrast to $\Delta Ent(\hat{\m \tau}_{i \mathcal{I}_s}, a, b)$, $DEMP(\hat{\m \tau}_{i \mathcal{I}_s}, a, b)$ function is not symmetric, and therefore, in general it is different to merge part $I_a$ to part $I_b$ than to merge part $I_b$ to part $I_a$. 

Similarly to the total Entropy, we have approximated the expected value of DEMP function
\[
\mathbb{E}\left[DEMP \left(
 \left( 
 \frac{\pi_a \hat{f}_{I_a}(\m x)}{\pi_a \hat{f}_{I_a}(\m x) + (1-\pi_a) \hat{f}_{I_b, \lambda}(\m x)}, 
 \frac{(1-\pi_a) \hat{f}_{I_b, \lambda}(\m x)}{\pi_a \hat{f}_{I_a}(\m x) + (1-\pi_a) \hat{f}_{I_b, \lambda}(\m x)} \right), a, b\right) | \m x \sim f \right].
\]

In Figure~\ref{fig:mu_varying} (top-right) DEMP is represented for $\lambda$, $0 \leq \lambda \leq 3$ and for $\pi_a \in \{ 0.1, 0.2, 0.3, 0.4, 0.5, 0.6, 0.7, 0.8, 0.9\}$. As commented, in this approach there is no symmetry between $a$ and $b$, we obtain different curves for different values of $\pi_a$. In this case, for $\lambda$ fixed, the higher measure of confusion is obtained when $\pi_a$ is lower. Another feature is that when $\lambda$ tends to zero the curves tends to $0$ if $\pi_a > 0.5$, to $1$ if $\pi_a < 0.5$ and to $0.5$ if $\pi_a = 0.5$. This is due to the fact that when $\hat{f}_{I_a}$ and $\hat{f}_{I_b, \lambda}$ are close enough, deciding if $\hat{\tau}_{i I_{b}}$ is higher to $\hat{\tau}_{i I_{a}}$ depends only on the value of the fixed $\pi_a$.

\subsection*{DEMP modified algorithm}

In DEMP approach, for an observation generated from a pdf $\hat{f}_{I_a}$, the role played by the indicator function in Equation~\ref{demp_criteria} is to count the number of times the observation is classified to part $I_b$, and therefore, to estimate the probability of classifying an observation to part $I_b$.

\cite{longford2014} proposes a similar approach. Instead of estimating the probability of classifying an observation to part $I_b$, they propose to estimate the probability that the osevation is generated by component $\hat{f}_{I_b}$ taken into an account that the observation has been generated by either $\hat{f}_{I_a}$ or $\hat{f}_{I_b}$. 

In \cite{longford2014} approach, the role played in Equation~\ref{demp_criteria} by the indicator function takes the form $\frac{\hat{\tau}_{iI_b}}{\hat{\tau}_{iI_a} + \hat{\tau}_{iI_b}}$. Although in  \cite{longford2014} the estimation $\frac{\hat{\tau}_{iI_b}}{\hat{\tau}_{iI_a} + \hat{\tau}_{iI_b}}$ is obtained by simulating a large amount of observation from component $\hat{f}_{I_a}$, similarly to the DEMP estimator, we can build a direct estimator

\begin{equation}\label{demp_criteria}
DEMP_2(\hat{\m \tau}_{i \mathcal{I}_s}, a, b) =\frac{ \sum_{i=1}^n \hat{\tau}_{iI_a} \hat{\tau}_{iI_b}(\hat{\tau}_{iI_a} + \hat{\tau}_{iI_b})^{-1}  }{\sum_{i=1}^n \hat{\tau}_{iI_a} }.
\end{equation}

Similar to DEMP method, the modified version of DEMP algorithm is also not symmetric. An important difference between $DEMP$ and $DEMP_2$ is that $DEMP_2$ only depends on parts  $iI_a$ and $iI_b$ whilest $DEMP$ method is affected by the other parts.

In Figure~\ref{fig:mu_varying} (bottom-left). when we have approximated the expected value of $DEMP_2$ function
\[
\mathbb{E}\left[DEMP_2 \left(
 \left( 
 \frac{\pi_a \hat{f}_{I_a}(\m x)}{\pi_a \hat{f}_{I_a}(\m x) + (1-\pi_a) \hat{f}_{I_b, \lambda}(\m x)}, 
 \frac{(1-\pi_a) \hat{f}_{I_b, \lambda}(\m x)}{\pi_a \hat{f}_{I_a}(\m x) + (1-\pi_a) \hat{f}_{I_b, \lambda}(\m x)} \right), a, b\right) | \m x \sim f \right],
\]
we have observed the behaviour is similar when function $\hat{f}_{I_a}$ and $\hat{f}_{I_b, \lambda}$ are distant, but when they are close, each curve tends to the value of $\pi_a$.


%
%The assimetry results in situation where merging $I_a$ to $I_b$ is suitable, but not the other way around. This concept
%
%
%
%The DEMP methods is similar to the method presented in \cite{longford2014}
%
%The function indicator $\mathbbm{1}_{\left[ \forall j\; \hat{\tau}_{i I_{b}} \geq \hat{\tau}_{iI_j} \right]}$ is defined by
%\[
%\mathbbm{1}_{\left[ \forall j\; \hat{\tau}_{i I_{b}} \geq \hat{\tau}_{iI_j} \right]} =
%\left\{\begin{array}{ll}	
%1 & \text{if $\forall j\; \hat{\tau}_{i I_{b}} \geq \hat{\tau}_{iI_j}$}\\
%0 & \text{if $\exists j\; \hat{\tau}_{i I_{b}} < \hat{\tau}_{iI_j}$}
%\end{array}\right.
%\]


%{\color{red} Martin diu que potser hauria de ser $a$ a sota.  Diria que no.}
%{\color{red} Dir què és el DEMP i com mesura la confusió entre b i a. En Hennig parla de misclassification}

\subsection*{The log-ratio algorithm}

In the total Entropy approach, the idea of confusion between components is build over the idea that closest the posterior probability vector are from $(\frac{1}{s}, \dots, \frac{1}{s})$ more confused are the components. In contrast. Instead, in DEMP approach the idea of confusion is build over the probability of classifying an observation to one component when the observation was generated from another component.

A completely different approach can be considered if the idea of confusion between part $I_a$ and $I_b$ is build by comparing the components $\hat{\tau}_{iI_a}$ and $\hat{\tau}_{iI_b}$ of the posterior probability vector $\hat{\m \tau}_{i \mathcal{I}_s}$. The approach is that if an observation is generated from probability density function $\hat{f}_{I_a}$ the higher is $\hat{\tau}_{iI_b}$ respect to $\hat{\tau}_{iI_a}$ the higher is the confusion between them. 

%For a partition $\mathcal{I}_s$, the dissimilarity measure $Ent( \hat{\m \tau}_{i \mathcal{I}_s} )$ is minimum when components of the posterior probabilities vector, $\hat{\m \tau}_{i \mathcal{I}}$, is the same, $\m \tau_0 = (\frac{1}{s}, ...,\frac{1}{s})$. The function $Ent( \hat{\m \tau}_{i \mathcal{I}_s} )$ grows as the probability vectors separates from the $\m \tau_0$.
%
%
%The total Entropy approach is an approach which takes into an account 

%The log-ratio approach, proposes the hierarchical combination of components $\mathcal{I}_1, \dots, \mathcal{I}_k$ defined as follows: starting from partition $\mathcal{I}_k = \{\{1\},\dots, \{k\}\}$ at each step the method combines two parts. If at current step we have the partition  $I_1, \dots, I_s$ the two parts, $I_a, I_b$, $1 \leq a,b \leq s$,  to be combined are those that \emph{minimise} log-ratio criterion by

To measure the relative differences between  $\hat{\tau}_{iI_b}$ and $\hat{\tau}_{iI_a}$, we propose to use the log ratio between them, i.e $log( \hat{\tau}_{iI_b}/\hat{\tau}_{iI_a})$. The approach proposes to combine those parts $I_a$ and $I_b$ maximising the following criteria, 

\[
LOG(\hat{\m \tau}_{i \mathcal{I}_s}, a, b) = \frac{\sum_{i=1}^n  \hat{\tau}_{iI_a}  \log( \frac{ \hat{\tau}_{iI_b} }{ \hat{\tau}_{iI_a} })}{\sum_{i=1}^n  \hat{\tau}_{iI_a} }.
\]

The distance between compositions, $d_{\mathcal{A}}(\m x,\m y)$ defined in \citep{aitchison1986statistical} is
\[
d_{\mathcal{A}}\left((\hat{\tau}_{iI_a}, \hat{\tau}_{iI_b}), (\frac{1}{2}, \frac{1}{2})\right) = \left| \log( \frac{ \hat{\tau}_{iI_b} }{ \hat{\tau}_{iI_a} }) \right|.
\]
If an observation comes from $\hat{f}_{I_a}$ is expected to have $\hat{\tau}_{iI_a} > \hat{\tau}_{iI_b}$, and therefore, similarly to the Entropy approach, using the distance $d_{\mathcal{A}}(\m x,\m y)$, the log-ratio approach is measuring how different is the posterior probability vector $(\hat{\tau}_{iI_a}, \hat{\tau}_{iI_b})$ of $(\frac{1}{2}, \frac{1}{2})$.

\begin{figure}[!h]
\centering
\includegraphics[scale=.6]{figure01/fig02.eps}
\caption{$\lambda$ varying {\color{red} Aquesta figura la treuria, no dona la noció de distancia que volem donar.}}
\label{fig:mu_varying_logit}
\end{figure}

The approximation of the expected value of $LOG$ function
\[
\mathbb{E}\left[LOG \left(
 \left( 
 \frac{\pi_a \hat{f}_{I_a}(\m x)}{\pi_a \hat{f}_{I_a}(\m x) + (1-\pi_a) \hat{f}_{I_b, \lambda}(\m x)}, 
 \frac{(1-\pi_a) \hat{f}_{I_b, \lambda}(\m x)}{\pi_a \hat{f}_{I_a}(\m x) + (1-\pi_a) \hat{f}_{I_b, \lambda}(\m x)} \right), a, b\right) | \m x \sim f \right],
\]
is shown in Figure~\ref{fig:mu_varying} (top-right). For $\lambda$ fixed, it shows a similar behaviour as $DEMP$ and $DEMP_2$, that is, lower the $\pi_a$ values, the higher the confusion is. In Figure~\ref{fig:mu_varying_logit} we have represented the inverse of the logit of the expected value.



%%\[
%%\frac{\sum_{i=1}^n \mathbbm{1}_{\left[ \forall j\; \hat{\tau}_{i I_{a}} \geq \hat{\tau}_{iI_j} \right]} \log( \frac{ \hat{\tau}_{iI_a} }{ \hat{\tau}_{iI_b} })}{\sum_{i=1}^n \mathbbm{1}_{\left[ \forall j\; \hat{\tau}_{i I_{a}} \geq \hat{\tau}_{iI_j} \right]}}.
%%\]
%
%
%
%{\color{red} Dir que la mesura és igual a la mitjana de les distàncies entre la subcomposició $(\tau_a, \tau_b)$ i $(0.5,0.5)$ per les observacions classificades a $a$.}
%
%\begin{itemize}
%\item a local measure of how possible is to confuse $I_b$ with $I_a$ depending on each observation $\m x_i$, $\log( \frac{ \hat{\tau}_{iI_a} }{ \hat{\tau}_{iI_b} })$, and
%\item a measure of how associated is an observation $\m x_i$ to cluster $I_a$, $\mathbbm{1}_{\left[ \forall j\; \hat{\tau}_{i I_{a}} \geq \hat{\tau}_{iI_j} \right]}$.
%\end{itemize}

\section{Unifying the approaches}
\label{confusion}

In previous section, we have presented different approaches to merge the components of a finite mixture. Some of them were approaches presented by other authors, and the others, up to our knowledge were presented for the first time. In this section, we consider the previous methods into a more general setting.

Consider a posterior probability vector $\hat{\m \tau}_{i \mathcal{I}_s}$. Let $\lambda(\hat{\m \tau}_{i \mathcal{I}_s}, a, b)$  be a function measuring the chances that an observation $\m x_i$ coming from $\hat{f}_{I_a}$ is classified to part $I_a$. In previous section we have seen different possibilities for this function:
\begin{itemize}
\item $\lambda(\hat{\m \tau}_{i \mathcal{I}_s}, a, b) = (\hat{\tau}_{iI_a}+\hat{\tau}_{iI_b}) \log(\hat{\tau}_{iI_a} + \hat{\tau}_{iI_b}) - \hat{\tau}_{iI_a} \log(\hat{\tau}_{iI_a}) + \hat{\tau}_{iI_b} \log(\hat{\tau}_{iI_b})$,
\item $\lambda(\hat{\m \tau}_{i \mathcal{I}_s}, a, b) =\mathbbm{1}\left( \forall j\; \hat{\tau}_{i I_{b}} \geq \hat{\tau}_{iI_j} \right)$, 
\item $\lambda(\hat{\m \tau}_{i \mathcal{I}_s}, a, b) =\hat{\tau}_{iI_b}(\hat{\tau}_{iI_a} + \hat{\tau}_{iI_b})^{-1}$, 
\item $\lambda(\hat{\m \tau}_{i \mathcal{I}_s}, a, b) =\log( \frac{ \hat{\tau}_{iI_b} }{ \hat{\tau}_{iI_a} })$.
\end{itemize}
Moreover, let $\omega(\hat{\m \tau}_{i \mathcal{I}_s}, a)$ be a function measuring the representativeness of observation $x_i$ of part $I_a$. For example:
\begin{itemize}
\item $\omega(\hat{\m \tau}_{i \mathcal{I}_s}, a) = const$,
\item $\omega(\hat{\m \tau}_{i \mathcal{I}_s}, a) =  \hat{\tau}_{iI_a}$ or
\item $\omega(\hat{\m \tau}_{i \mathcal{I}_s}, a) = \mathbbm{1}\left( \forall j\; \hat{\tau}_{i I_{a}} \geq \hat{\tau}_{iI_j} \right)$.
\end{itemize}
In the first example, each observation represents part $I_a$ equally. In the last example, only observations classified to part $I_a$ represent the element of part $I_a$.

With functions $\lambda(\hat{\m \tau}_{i \mathcal{I}_s}, a, b)$ and $\omega(\hat{\m \tau}_{i \mathcal{I}_s}, a)$ we can consider the following approach. For a partition $\mathcal{I}_s = \{ I_1, \dots, I_s\}$ let $\hat{\m\tau}_{i \mathcal{I}_s} = \left( \hat{\tau}_{i I_1} , \dots, \hat{\tau}_{i I_s}  \right)$ for $i$, $1 \leq i \leq n$, be the posterior probability vectors calculated from sample $X$. With the set $\hat{\m\tau}_{i \mathcal{I}_s}$ consider the following approach. Given partition $\mathcal{I}_s$ merge the two parts $I_a$ and $I_b$, into one part $I_a \cup I_b$ which maximise
\begin{equation}\label{unifying_equation}
\frac{\sum_{i=1}^n \omega(\hat{\tau}_{i \mathcal{I}_s}, I_a) \lambda(\hat{\tau}_{i \mathcal{I}_s}, I_a, I_b)}{\sum_{i=1}^n \omega(\hat{\tau}_{i \mathcal{I}_s}, I_a) }.
\end{equation}

%In this section, we extend the ideas that motivate the log-ratio algorithm to a general framework. We consider a local measure given by function $\lambda(\hat{\m\tau}_{i \mathcal{I}}, I, J)$ of how similar are the clusters given by part $I$ and part $J$ in $\m x_i$, and a measure of how associated is an observation $\m x_i$ to the cluster given by part $I$, $\omega(\m x_i, I)$. This general approach proposes the hierarchical combination of components $\mathcal{I}_1 \dots, \mathcal{I}_k$ defined as follows: starting from partition $\mathcal{I}_k = \{\{1\},\dots, \{k\}\}$ at each step the method combine two parts. If at current step we have the partition  $I_1, \dots, I_s$ the two parts, $I_a, I_b$, $1 \leq a,b \leq s$,  to be combined are those that \emph{minimise} the general criterion given by


The Equation~\ref{unifying_equation} allows to model all the approaches given in previous section,
\begin{itemize}
\item Entropy
\begin{itemize}
\item $\omega(\hat{\m \tau}_{i \mathcal{I}_s}, a) = const$
\item $\lambda(\hat{\m \tau}_{i \mathcal{I}_s}, a, b) = (\hat{\tau}_{iI_a}+\hat{\tau}_{iI_b}) \log(\hat{\tau}_{iI_a} + \hat{\tau}_{iI_b}) - \hat{\tau}_{iI_a} \log(\hat{\tau}_{iI_a}) + \hat{\tau}_{iI_b} \log(\hat{\tau}_{iI_b})$
\end{itemize}

\item $DEMP$
\begin{itemize}
\item $\omega(\hat{\m \tau}_{i \mathcal{I}_s}, a) = \hat{\tau}_{iI_a}$
\item $\lambda(\hat{\m \tau}_{i \mathcal{I}_s}, a, b) = \mathbbm{1}\left( \forall j\; \hat{\tau}_{i I_{b}} \geq \hat{\tau}_{iI_j} \right)$
\end{itemize}

\item $DEMP_2$
\begin{itemize}
\item $\omega(\hat{\m \tau}_{i \mathcal{I}_s}, a) = \hat{\tau}_{iI_a}$
\item $\lambda(\hat{\m \tau}_{i \mathcal{I}_s}, a, b) = \hat{\tau}_{iI_b}(\hat{\tau}_{iI_a} + \hat{\tau}_{iI_b})^{-1}$
\end{itemize}

\item $Log$
\begin{itemize}
\item $\omega(\hat{\m \tau}_{i \mathcal{I}_s}, a) = \hat{\tau}_{iI_a}$
\item $\lambda(\hat{\m \tau}_{i \mathcal{I}_s}, a, b) = \log( \frac{ \hat{\tau}_{iI_b} }{ \hat{\tau}_{iI_a} })$
\end{itemize}
\end{itemize}



{\color{red} Aquest paràgraf està molt "xulo". Però crec que cal haver-ne parlat abans quan s'ha presentat els índexos. 
Vull dir, quan es presetna cada índex aniria bé explicar la idea "geomètrica" que ha pensat l'autor (Henning,  Baudry, Comas). Després aquí, aquest aparta xulo s'entendrà molt més. }


The three algorithms(Entropy, DEMP, Log) give us three different possible definition for functions $\lambda(\hat{\tau}_{i \mathcal{I}_s}, I_a, I_b)$ and $\omega(\hat{\tau}_{i \mathcal{I}_s}, I_a)$. For the function $\lambda(\hat{\tau}_{i \mathcal{I}_s}, I_a, I_b)$ the entropy algorithm considers that the more the confusion is between two parts the higher is the difference between the entropy considering the partition with the two part together and merged. The way the confusion is defined by the DEMP algorithm is more subtle, it states that the higher the confusion between two parts the higher the number of observations not classified to one part. The log-ratio approach relates the confusion respect to the magnitude of the log-ratio. For the function $\omega(\hat{\tau}_{i \mathcal{I}_s}, I_a)$, the entropy algorithm weigths equally the local information of each observation to decide if $I_a$ should be merged to $I_b$. By contrast, the DEMP and Log algorithm weigths the observation according to the probability of pertinence to $I_a$. 
%Finally, the log-ratio algorithm weigths with respect the final pertinence to $I_a$.

\begin{table}[!ht]
\begin{tabular}{c  c | >{\centering}m{0.7in} | >{\centering}m{0.7in} | >{\centering}m{0.7in} | m{0in}}
 & \multicolumn{1}{c}{} & \multicolumn{3}{c}{$\omega(\boldsymbol\tau_i, a)$} &\\

 & \multicolumn{1}{c}{} & \multicolumn{1}{c}{} & \multicolumn{1}{c}{} & \multicolumn{1}{c}{} & \multicolumn{1}{c}{}\\

 & \multicolumn{1}{c}{} & \multicolumn{1}{c}{1} & \multicolumn{1}{c}{$\tau_{ia}$} & \multicolumn{1}{c}{$\mathlarger{\mathbbm{1}}\left\{  \forall \ell\; \; \tau_{ia} \geq \tau_{i\ell}  \right\}$} &\\ \cline{3-5} 

& $\large\substack{(\tau_{i a}+\tau_{i b}) \log(\tau_{i a} + \tau_{i b}) - \\ - \tau_{i a} \log(\tau_{i a}) - \tau_{i b} \log(\tau_{i b}) }$ & $Entropy$ &  &  &\\[5em] \cline{3-5}

\rotatebox[origin=c]{90}{$\lambda(\boldsymbol\tau_i, a, b)$} & $\mathlarger{\mathbbm{1}}\left\{  \forall \ell \; \tau_{hb} \geq \tau_{h\ell}  \right\}$ & & $DEMP$  & -- & \\[5em] \cline{3-5}

& ${\tau_{i b}}({\tau_{i a}+\tau_{i b}})^{-1}$ &  &  $DEMP_2$ &  &\\[5em] \cline{3-5}

& $\log{\tau_{i b} / \tau_{i a}}$ & & $Log$ &   &\\[5em] \cline{3-5}

& $\tau_{i b}$ &  &  &  &\\[5em] \cline{3-5}

\end{tabular}
\caption{Different combinations of score functions}
\end{table}

{\color{red} Seguir l'exemple de les 6 components i estudiar com es calculen els tres indexos en un determinat pas de la jerarquia. Per exemple, quan es decideix fusion el {3} amb el {1,6}. O alguna cosa simila. Preparar-ho pel CoDaWor juntament amb un cas d'estudi final}

{\color{red} Posar lo de Kullback Leibler prenent tau i logratio. Donar la idea que no només és una unificació de tres index si no que una generalització.}

\section{Simulation}

\begin{itemize}
\item For a fixed value of $\omega$, we repeat $1000$ the following process:
\begin{itemize}
\item Generate a sample of size $n=500$ with $D=5$ coordinates coming from the gaussian mixture with $3$ components whose components have maximum overlapping $\omega$.
\item Fit a mixture
\end{itemize}


\end{itemize}
%\section{Fitting a hierarchical model  to a mixture of gaussian distributions}

%Using Equation~\ref{unifying_equation} different methods can be proposed.

\begin{figure}[!h]
\centering
\includegraphics[width=\textwidth]{figure02/boxplot.pdf}
\caption{$\lambda$ varying {\color{red} Aquesta figura la treuria, no dona la noció de distancia que volem donar.}}
\label{fig:mu_varying_logit}
\end{figure}

\bibliographystyle{apalike}
\bibliography{tex/combining_mixtures}{}

\end{document}
