\documentclass{letter}
\usepackage{hyperref}
\usepackage[utf8]{inputenc}


%\address{Department of Computer Science, Applied Mathematics and Statistics,\\ Univesitat de Girona, P4-Campus Montilivi,\\ 17071 Girona, Spain}
\begin{document}


\begin{letter}{
SORT Statistics and Operations Research Transactions\\
Institut d'Estadística de Catalunya (Idescat)\\
Via Laietana, 58 — 08003 Barcelona. SPAIN\\
Tel. +34-93.557.30.76 — Fax +34-93.557.30.01}
\opening{Dear Editor:}

Please find enclosed the manuscript: “Logratio methods in mixture models for compositional data sets”, by Marc Comas-Cufí, Josep-Antoni Martín-Fernández, Glòria Mateu-Figueras, to be submitted as a Regular Paper to the SORT journal for consideration for publication. All co-authors have seen and agree with the contents of the manuscript and there is no financial interest to report. We certify that the submission is original work and is not under review at any other publication.

In this manuscript we show that a finite mixture of distributions defined on the real space is not appropriate for compositional data. Traditional methods we can find in the literature are revisited and the major difficulties are detailed. A new and general proposal using a mixture of distributions defined on log-ratio coordinates is presented. Finally, to compare the different methodologies, different finite mixture distributions are adjusted to a real compositional dataset.

We believe that our findings could be of interest to the readers of SORT journal because they bring a new methodology for modelling compositional data through the use of finite mixture distributions, which may be of application in many fields, such as life or social sciences. 

We hope that the editorial board will agree on the interest of this study.

Sincerely yours,

\bigskip

Marc Comas-Cufí on behalf of the authors.

\noindent\rule{8cm}{0.4pt}

Address of the corresponding author: \\
Marc Comas-Cufí,
Dept. Computer Science, Applied Mathematics and Statistics, University of Girona, Campus Montilivi (P4), E-17071 Girona, Spain. (mcomas@imae.udg.edu; (+34) 972 418426)

\end{letter}

\end{document}